\documentclass{sigchi}

% Use this section to set the ACM copyright statement (e.g. for
% preprints).  Consult the conference website for the camera-ready
% copyright statement.

% Copyright
%\CopyrightYear{2020}
%\setcopyright{acmcopyright}
%\setcopyright{acmlicensed}
%\setcopyright{rightsretained}
%\setcopyright{usgov}
%\setcopyright{usgovmixed}
%\setcopyright{cagov}
%\setcopyright{cagovmixed}
% DOI
%\doi{https://doi.org/10.1145/3313831.XXXXXXX}
% ISBN
% \isbn{978-1-4503-6708-0/20/04}
%Conference
\conferenceinfo{CHI'20,}{April  25--30, 2020, Honolulu, HI, USA}
%Price

% Use this command to override the default ACM copyright statement
% (e.g. for preprints).  Consult the conference website for the
% camera-ready copyright statement.

%% HOW TO OVERRIDE THE DEFAULT COPYRIGHT STRIP --
%% Please note you need to make sure the copy for your specific
%% license is used here!
% \toappear{
% Permission to make digital or hard copies of all or part of this work
% for personal or classroom use is granted without fee provided that
% copies are not made or distributed for profit or commercial advantage
% and that copies bear this notice and the full citation on the first
% page. Copyrights for components of this work owned by others than ACM
% must be honored. Abstracting with credit is permitted. To copy
% otherwise, or republish, to post on servers or to redistribute to
% lists, requires prior specific permission and/or a fee. Request
% permissions from \href{mailto:Permissions@acm.org}{Permissions@acm.org}. \\
% \emph{CHI '16},  May 07--12, 2016, San Jose, CA, USA \\
% ACM xxx-x-xxxx-xxxx-x/xx/xx\ldots \$15.00 \\
% DOI: \url{http://dx.doi.org/xx.xxxx/xxxxxxx.xxxxxxx}
% }

% Arabic page numbers for submission.  Remove this line to eliminate
% page numbers for the camera ready copy
% \pagenumbering{arabic}

% Load basic packages
\usepackage{balance}       % to better equalize the last page
\usepackage{graphics}      % for EPS, load graphicx instead 
\usepackage[T1]{fontenc}   % for umlauts and other diaeresis
\usepackage{txfonts}
\usepackage{mathptmx}
\usepackage[pdflang={en-US},pdftex]{hyperref}
\usepackage{color}
\usepackage{booktabs}
\usepackage{textcomp}


% Some optional stuff you might like/need.
\usepackage{microtype}        % Improved Tracking and Kerning
% \usepackage[all]{hypcap}    % Fixes bug in hyperref caption linking
\usepackage{ccicons}          % Cite your images correctly!
% \usepackage[utf8]{inputenc} % for a UTF8 editor only

% If you want to use todo notes, marginpars etc. during creation of
% your draft document, you have to enable the "chi_draft" option for
% the document class. To do this, change the very first line to:
% "\documentclass[chi_draft]{sigchi}". You can then place todo notes
% by using the "\todo{...}"  command. Make sure to disable the draft
% option again before submitting your final document.
\usepackage{todonotes}

% Paper metadata (use plain text, for PDF inclusion and later
% re-using, if desired).  Use \emtpyauthor when submitting for review
% so you remain anonymous.
\def\plaintitle{Project Spellda}
\def\plainauthor{James Nail, Logan Brown, Steven Gray}
\def\emptyauthor{}
\def\plainkeywords{Authors' choice; of terms; separated; by
  semicolons; include commas, within terms only; this section is required.}
\def\plaingeneralterms{Documentation, Standardization}

% llt: Define a global style for URLs, rather that the default one
\makeatletter
\def\url@leostyle{%
  \@ifundefined{selectfont}{
    \def\UrlFont{\sf}
  }{
    \def\UrlFont{\small\bf\ttfamily}
  }}
\makeatother
\urlstyle{leo}

% To make various LaTeX processors do the right thing with page size.
\def\pprw{8.5in}
\def\pprh{11in}
\special{papersize=\pprw,\pprh}
\setlength{\paperwidth}{\pprw}
\setlength{\paperheight}{\pprh}
\setlength{\pdfpagewidth}{\pprw}
\setlength{\pdfpageheight}{\pprh}

% Make sure hyperref comes last of your loaded packages, to give it a
% fighting chance of not being over-written, since its job is to
% redefine many LaTeX commands.
\definecolor{linkColor}{RGB}{6,125,233}
\hypersetup{%
  pdftitle={\plaintitle},
% Use \plainauthor for final version.
%  pdfauthor={\plainauthor},
  pdfauthor={\emptyauthor},
  pdfkeywords={\plainkeywords},
  pdfdisplaydoctitle=true, % For Accessibility
  bookmarksnumbered,
  pdfstartview={FitH},
  colorlinks,
  citecolor=black,
  filecolor=black,
  linkcolor=black,
  urlcolor=linkColor,
  breaklinks=true,
  hypertexnames=false
}

% create a shortcut to typeset table headings
% \newcommand\tabhead[1]{\small\textbf{#1}}

% End of preamble. Here it comes the document.
\begin{document}

\title{\plaintitle}

\numberofauthors{3}
\author{%
  \alignauthor{James Nail\\
    \affaddr{City, Country}\\
    \email{e-mail address}}\\
  \alignauthor{Logan Brown\\
    \affaddr{City, Country}\\
    \email{e-mail address}}\\
  \alignauthor{Steven Gray\\
    \affaddr{City, Country}\\
    \email{e-mail address}}\\
}

\maketitle

\begin{abstract}
Our project is Project Spellda, a two-dimensional, top down role playing game developed in Godot. Our game will minimally include 3 different levels: a tutorial level to introduce the mechanics of the game, followed by an overworld and dungeon that are intertwined. At the beginning of the game, the player will choose two of the four in-game elements: Fire, Water, Air, and Earth. The player then explores the surrounding area, looking for upgrades to their default spells. At any time, the player can dip into the dungeon to test their abilities on the stronger enemies that await them, with the end goal being to get to the center of the dungeon and slay the boss. The game is made entirely using Godot's built-in scripting language GDScript.
\end{abstract}


% Author Keywords
\keywords{\plainkeywords}



\section{Introduction}

Project Spellda is a top-down 2D arcade style RPG game, mixing elements from the likes of Legend of Zelda among other dungeon crawlers
and RPGs of the era.  The game will have an overworld with a couple different enemy types, along with several offshoot dungeon-like areas
and one main dungeon existing under the overworld.  The dungeon will be relatively harder than the overworld, with better loot and harder
bosses, as well as mechanics used in the dungeon-like offshoots found scattered in the overworld.  The overworld will have varied difficulties
across numerous zones, with several of these difficulties tying directly to difficulties found within the dungeon.\\\\
	
The main mechanic of the game is the spell system.  Players have access to four basic spells, but can only use two at any given time.  
These spells can be upgraded with items found throughout the world with effects such as triple shot, rapid cast, area of effect, and high 
damage.  The more powerful items have downsides such as decreased casting speed, lower damage, among other post-casting debuffs.  The spells
will have a "rock-paper-scissors" system, where one spell will counter one, but will be weak to another.  Thanks to having four spells and
the player only having two slots, this makes sure the player will counter exactly two types, be weak to at least one, and be neutral against
at least one at all times.

\section{Technical Specifications}

We are using the Godot game engine\cite{godot}, with much of our scripting done in Python.  We have not determined exactly how we will handle spriting yet.

Some of our knowledge came from a wonderful playlist on youtube (from the youtuber "jmbiv"\cite{jmbiv}), showing how to do certain things so we could translate that knowledge to our project.

\bibliographystyle{plain}
\bibliography{bibliography}

\end{document}
