\documentclass[12pt]{article}
\title{Project Spellda}
\author{James Nail, Logan Brown, and Steven Gray}
\usepackage{hyperref}
\begin{document}
	\maketitle
\newpage
	\section*{Abstract}
	This project is a two-dimensional role playing game developed in Godot. The outline of the game includes a minimum of 3 different
	levels, a dungeon to explore, an over-world to explore, and tutorial level to introduce the main mechanic. The main mechanic of 
	the game is the player has access to a subset of the four main elements: Fire, Water, Air, and Earth. Whichever elements the
	player has access to at any time can be modified by "techniques" found around the world. These could do a variety of things
	to your chosen elements, such as: AoE (Area of Attack), Rapid Fire, DoT (Damage over time), etc...
	Our stretch goals include adding a secret boss to fight, as well as adding "elemental enemies", or enemy types that
	have limited magical capabilities. Farther stretch goals include add more elements that the player can access for
	variety in overall game-play and added replay value.
\newpage
	\section*{Introduction}
	Project Spellda is a top-down 2D arcade style RPG game, mixing elements from the likes of Legend of Zelda among other dungeon crawlers
	and RPGs of the era.  The game will have an overworld with a couple different enemy types, along with several offshoot dungeon-like areas
	and one main dungeon existing under the overworld.  The dungeon will be relatively harder than the overworld, with better loot and harder
	bosses, as well as mechanics used in the dungeon-like offshoots found scattered in the overworld.  The overworld will have varied difficulties
	across numerous zones, with several of these difficulties tying directly to difficulties found within the dungeon.\\\\
	
	The main mechanic of the game is the spell system.  Players have access to four basic spells, but can only use two at any given time.  
	These spells can be upgraded with items found throughout the world with effects such as triple shot, rapid cast, area of effect, and high 
	damage.  The more powerful items have downsides such as decreased casting speed, lower damage, among other post-casting debuffs.  The spells
	will have a "rock-paper-scissors" system, where one spell will counter one, but will be weak to another.  Thanks to having four spells and
	the player only having two slots, this makes sure the player will counter exactly two types, be weak to at least one, and be neutral against
	at least one at all times.
	
\newpage
	\section*{Technical Specifications}
	We are using the Godot game engine\cite{godot}, with much of our scripting done in Python.  We have not determined exactly how we will handle spriting yet.

	Some of our knowledge came from a wonderful playlist on youtube (from the youtuber "jmbiv"\cite{jmbiv}, showing how to do certain things so we could translate that knowledge to our project.


\bibliographystyle{plain}
\bibliography{bibliography}

\end{document}
