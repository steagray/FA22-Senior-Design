\documentclass[xcolor=table]{beamer}

\usetheme[secheader,compress]{Madrid} %Primary theme

\usepackage{verbatim}
\usepackage{graphicx}

%% UTM Colors
\definecolor{UTMblue}{rgb}{0.043137, 0.137254, 0.254901}
\definecolor{UTMorange}{rgb}{1.0, 0.509803, 0}

\setbeamercolor{palette primary}{bg=UTMblue,fg=white}
\setbeamercolor{palette secondary}{bg=UTMblue,fg=white}
\setbeamercolor{palette tertiary}{bg=UTMblue,fg=white}
\setbeamercolor{palette quaternary}{bg=UTMblue,fg=white}
\setbeamercolor{structure}{fg=UTMblue} % itemize, enumerate, etc
\setbeamercolor{section in toc}{fg=UTMblue} % TOC sections
\setbeamercolor{title}{fg=UTMorange}

\setbeamercolor{subsection in head/foot}{bg=UTMorange,fg=white}

%%%%%%%%%%% BEGIN MACROS %%%%%%%%%%%%%%%%%%
% frameT: Frame with title
\newcommand{\frameT}[2]{\frame{\frametitle{#1} #2}}

% frameF: Fragile frame with title
\newcommand{\frameF}[2]{
  \begin{frame}[fragile]
    \frametitle{#1}
    #2
  \end{frame}
}

% frameTop: Frame aligned t the top
\newcommand{\frameTop}[2]{\frame[t]{\frametitle{#1} #2}}


\newcommand{\tab}{\hspace{1cm}}

\newcommand{\spaceor}{\hspace{5pt} \textbf{or} \hspace{5pt}}

%%%%%%%%%%% END MACROS %%%%%%%%%%%%%%%%%%%%



\begin{document}

\title{Project Spellda}

\author{Steven Gray, James Nail, and Logan Brown}
\institute{UT-Martin}
\date{\today}

%%%%%%%%%%% BEGIN TITLE %%%%%%%%%%%%%%%%%%
\frame{\titlepage}

 %\section{Outline}
%%%%%%%%%%%% END TITLE  %%%%%%%%%%%%%%%%%%


\section{Introduction}
\frameT{Motivation} {
	Why a video game?
	\bigskip
  \begin{enumerate}
    \item Wanted to try something new to each of us
      \bigskip
    \item Wanted to do something we were all interested in
  \end{enumerate}

  \bigskip
}

\frameT{Game Overview} {
	\begin{enumerate}
		\item 3 Levels: Tutorial, Overworld, and Dungeon
		\bigskip
		\item 4 different enemies, 2 melee and 2 ranged
		\bigskip
		\item 4 different elements to control
		\bigskip
		\item Various puzzles and offshoots to incentivize exploration
	\end{enumerate}
}

\section{Development}

\frameT{Engine} {
	\begin{center}
		Godot/GDScript
		\bigskip
	\end{center}
	
	\begin{center}
		\includegraphics[width=.7\linewidth]{figures/godotscreenshot.png}
	\end{center}
}

\frameT{Inspirations} {
	\begin{figure}[ht]
    \begin{minipage}[b]{0.53\linewidth}
			\begin{itemize}
				\item Aesthetic:
				\item Thomas was Alone
				\item Illumine
					\bigskip
				\item Mechanics:
				\item The Legend of Zelda (NES)
			\end{itemize}
    \end{minipage}
    \begin{minipage}[b]{0.4\linewidth}
				%  ------- post pictures here --------
			\includegraphics[width=.7\linewidth]{figures/illumine.png}
			\includegraphics[width=.7\linewidth]{figures/zelda.png}
			\includegraphics[width=.7\linewidth]{figures/thomas.jpg}
    \end{minipage}
  \end{figure}
}

\frameT{Demo} {
\begin{center}
	Here is a look into what we have!
\end{center}

\bigskip

\begin{center}
	\includegraphics[width=.7\linewidth]{figures/project.png}
\end{center}
}

\frameT{Changes from Initial Design} {
	% TURN THIS INTO A COUPLE SLIDES
	Over the course of development, our focus ended up shifting
	\begin{itemize}
		\item Expanded focus on dungeon and puzzle design
		\bigskip
		\item More focus on aesthetics and lighting
	\end{itemize}
}

\section{Comparisons}
\frameT{Comparison} {
	\begin{itemize}
		\item Lighting and shadows
	\end{itemize}
	\begin{figure}[ht]
    \begin{minipage}[b]{0.53\linewidth}
      \centering
			\includegraphics[width=\linewidth]{figures/earlyproj.png}
    \end{minipage}
    \hspace{0.5cm}
    \begin{minipage}[b]{0.37\linewidth}
      \centering
			\includegraphics[width=\linewidth]{figures/ourgame.png}
    \end{minipage}
  \end{figure}
}

\frameT{Lighting and Shadows} {
	\begin{itemize}
		\item Originally very bland and basic
	\end{itemize}
	\begin{center}
		\includegraphics[scale=0.25]{figures/earlyproj.png}
	\end{center}
}

\frameT{Lighting and Shadows} {
	\begin{itemize}
		\item Evolved to have much deeper visual design
	\end{itemize}
	\begin{center}
		\includegraphics[scale=0.25]{figures/ourgame.png}
	\end{center}
}

\frameT{Particles} {
	\begin{itemize}
		\item Started with no particles
		\item Got particles such as door particles, plate particles, etc.
	\end{itemize}
	\begin{center}
		\includegraphics{figures/particles.png}
	\end{center}
}

\section{Level Design}
\frameT{Tutorial}{
	\includegraphics[scale=0.25]{figures/tut_mockup.png}
}

\frameT{Overworld} {
	\includegraphics[scale=0.25]{figures/overworld.png}
}

\frameT{Side Area} {
	\includegraphics[scale=0.25]{figures/designexample.png}
}

\frameT{Dungeon} {
	
}

\section{Development}
\frameT{The Troubles} {
	\begin{enumerate}
		\item Movement
			\bigskip
		\item Lighting and Tiles
			\bigskip
		\item Team, not Group
			\bigskip
		\item Organization
	\end{enumerate}
}

\frameT{Future Work}{
	\begin{enumerate}
		\item Narrative
		\bigskip
		\item Secret Boss(/es)
		\bigskip
		\item More dungeons and an Expanded Overworld
		\bigskip
		\item Elemental weaknesses/resistences
	\end{enumerate}
}

\section{Conclusion}
\frameT{Results} {
	\begin{itemize}
		\item What did we get from this project?
			\bigskip
		\item Where do we go from here?
	\end{itemize}
}

\frameT{Thank You} {
  
	\begin{center}
		Contact Us!
	\end{center}

  \begin{center}
    Questions?
  \end{center}
  
  \begin{center}
    Comments?
  \end{center}


  \bigskip
  
  \begin{center}
  Further project/author information:
  \end{center}  

  \begin{center}
    https://github.com/steagray/FA22-Senior-Design
  \end{center}

  \begin{center}
		\begin{itemize}
			\item Steven Gray (steagray@ut.utm.edu)
			\item James Nail (jamgnail@ut.utm.edu)
			\item Logan Brown (logmbrow@ut.utm.edu)
		\end{itemize}
  \end{center}
}

\end{document}
